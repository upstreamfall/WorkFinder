\documentclass[11pt,a4paper]{article}

\usepackage[english,polish]{babel}
\usepackage[utf8]{inputenc}
\usepackage[T1]{fontenc}
\usepackage{lmodern}
\usepackage{indentfirst}
\usepackage{fullpage}
\usepackage{enumerate}
\usepackage{secdot}
\usepackage{verbatim}
\usepackage{graphicx}
\usepackage{url}
\usepackage{hyperref}
\usepackage[colorinlistoftodos,prependcaption,textsize=tiny]{todonotes}

\selectlanguage{polish}
\frenchspacing

\author{
	Paweł Bielicki\\
	Jakub Dutkowski\\
	Tomasz Janiszewski
}
\title{
	\huge{WorkFinder jako system poszukiwania pracy oparty na poczcie pantoflowej z wykorzystaniem technologii agentowych}\\
	\LARGE{Specyfikacja wstępna}
 } 	

\begin{document}
\maketitle

\listoftodos
\newpage


\section{Cel projektu}
Systemy informatyczne współczesnych organizacji stają się układami otwartymi, których głównym zadaniem jest zautomatyzowana komunikacja z partnerami biznesowymi, klientami i instytucjami. Rodzi to zapotrzebowanie na aplikacje będące w stanie wyręczyć użytkownika w wielu czynnościach, pozostawiając mu tylko podjęcie niezbędnych decyzji.

Jedną z czynności, dla której możliwe jest odciążenie użytkownika jest procedura poszukiwania pracy. Celem projektu będzie stworzenie wirtualnego środowiska wymiany ofert pracy umożliwiającego firmom znalezienie pracownika o odpowiednich kwalifikacjach, a z drugiej strony ułatwiającego użytkownikom poszukiwania pracodawcy odpowiadającego ich potrzebom.

Wynikiem projektu będzie w pełni funkcjonalny system mobilny złożony z wielu autonomicznych agentów, którzy będą reprezentować specjalistów oraz instytucje ich poszukujące. Będziemy rozpatrywać otoczenie branży IT.


\section{Opis problemu}
Poszukiwanie pracy niesie ze sobą wiele powtarzalnych i żmudnych czynności, które można powierzyć inteligentnemu agentowi, którego zadaniem będzie odszukanie dostępnych ofert, przefiltrowanie i przekazanie użytkownikowi tylko tych, które są zgodne z podanymi przez niego oczekiwaniami.
\\
Problem będziemy rozpatrywać z dwóch stron:
\begin{enumerate}
	\item poszukiwania pracy przez programistę,
	\item poszukiwania specjalistów spełniających kryteria rekrutacyjne.
\end{enumerate}

System ogłaszania ofert będzie oparty na tzw. `poczcie pantoflowej' \footnote{\url{http://pl.wiktionary.org/wiki/poczta_pantoflowa}} wewnątrz sieci kontaktów oraz na tzw. `tablicy', czyli miejscu, do którego wszyscy mają dostęp. Oferty pracy będą przekazywane do pracowników i subskrybentów, którzy będą mogli przekazać ofertę innym agentom ze swojego otoczenia.


\subsection{Poszukiwanie pracy przez programistę}
Pierwszą grupą użytkowników końcowych, do których adresowane jest to rozwiązanie są programiści poszukujący pracy.

By dobrać dobrze dopasowaną ofertę, czyli taką, której wymagania pokrywają się w dużej części z umiejętnościami kandydata, programista będzie musiał wypełnić ankietę posiadanych umięjętności oraz zdobytego doświadczenia zawodowego \todo{nice to have: integracja z zewnętrznym serwisem np. LinkedIn}. Każda umiejętności będzie posiadała ocenę z zakresu 1-5 świadczącą o poziomie zaawansowania, dodatkowo będzie możliwość podlinkowania projektu potwierdzającego umiejętność.

System będzie udostępniał każdemu z użytkowników interfejs webowy, w którym będzie mógł w przystępny sposób przeglądać listę dostępnych ofert pracy przygotowaną w oparciu o zdefiniowane przez niego preferencje.

Proces przygotowywania ofert będzie odbywał się w oparciu o komunikację P2P \footnote{(ang. \emph{Peer-to-Peer}) \url{http://pl.wikipedia.org/wiki/Peer-to-peer}} między autonomicznymi agentami, bez centralnego zarządzania.

Programista będzie mógł zbudować własną siatkę kontaktów, w której agenci będą wymieniali się aktualnymi ofertami. Moduł ten będzie brany pod uwagę w doborze oferty, np. osoba, którą mamy w kontaktach będzie mogła napisać krótką recenzję na temat pracodawcy, czy stanowiska. Możliwe będzie również wysyłanie grupie znajomych polecanych ofert.


\subsection{Poszukiwanie specjalistów}
Moduł ten będzie aplikacją dostępną dla firm poszukujących programistów o odpowiednim profilu.

Pracownicy działu HR będą mogli tworzyć profile o pożądanych przez ich firmy umiejętnościach i doświadczeniu, a także dołączać do oferty informacje o proponowanym stanowisku oraz oferowanych bonusach (jak karta multiSport, czy wynagrodzenie).

Oferty pracy będą ogłaszane na tzw. rynku pracy \todo{propozycja: agent w postaci servisów restowych, który będzie wyszukiwał oferty z miejsca `rynek' z podanymi parametrami} oraz przesyłane w formie biuletynu pracownikom.

W module dla pracodawcy będzie możliwe również ustalenie automatycznych etapów rekrutacji, np. wysłanie propozycji pracy, porównanie umiejętności, wysłanie i sprawdzenie testu teoretycznego/praktycznego.


\section{Techniczny opis systemu}
\todo{ten pkt. wydaje mi się niepotrzeby, uściślił bym go później}
System będzie stworzony przy pomocy technologii agentowej. Środowiskiem `życia' agenta będzie platforma JADE \todo{wstępnie}. Dostęp do platformy odbywać się będzie przez aplikację webową.

Projekt nie zakłada implementacji aplikacji dedykowanej na urządzenia mobilne, głównym powodem takiej decyzji jest specyfika rozpatrywanego problemu, poszukiwanie pracy wiąże się z conajwyżej kilkukrotnym logowaniem do platformy w ciągu dnia, więc nie jest wymagany stały nadzór. Plusem tego rozwiązania jest możliwość korzystania na wielu urządzeniach (komputer, tablet, telefon) oraz oszczędzenie zasobów urządzeń mobilnych.


\section{Koncepcja eksperymentu- scenariusze testowe}
W celu potwierdzenia poprawności działania systemu zostaną stworzone scenariusze testowe, które będą obrazować działanie systemu. Zostaną nakreślone proste scenariusze potwierdzające poszczególne funkcjonalności jak i pożądanie działanie systemu w rozbudowanych warunkach.

Scenariusze zostaną przedstawione jako funkcje IN/OUT, które będą przyjmować na wejściu stan środowiska: dostępnych użytkowników, oferty pracy, a na wyjściu oczekiwany rezultat, jak: programista zdobył pracę, lub oferta została rozesłana do wszystkich pracowników.
	
\end{document}
