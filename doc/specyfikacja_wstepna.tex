\documentclass[11pt,a4paper]{article}

\usepackage[english,polish]{babel}
\usepackage[utf8]{inputenc}
\usepackage[T1]{fontenc}
\usepackage{lmodern}
\usepackage{indentfirst}
\usepackage{fullpage}
\usepackage{enumerate}
\usepackage{secdot}
\usepackage{verbatim}
\usepackage{graphicx}

\selectlanguage{polish}
\frenchspacing

\author{
	Paweł Bielicki\\
	Jakub Dutkowski\\
	Tomasz Janiszewski
}
\title{
	\huge{WorkFinder jako system poszukiwania pracy oparty na poczcie pantoflowej z wykorzystaniem technologii agentowych}\\
	\LARGE{Specyfikacja wstępna}
 } 	

\begin{document}
\maketitle
\newpage

\section{Cel projektu}
Systemy informatyczne współczesnych organizacji stają się układami otwartymi, których głównym zadaniem jest zautomatyzowana komunikacja z partnerami biznesowymi, klientami i instytucjami. Rodzi to zapotrzebowanie na aplikacje będące w stanie wyręczyć użytkownika w wielu czynnościach, pozostawiając mu tylko podjęcie niezbędnych decyzji.
Jedną z czynności, dla której możliwe jest odciążenie użytkownika jest procedura poszukiwania pracy. Celem projektu będzie stworzenie wirtualnego środowiska wymiany ofert pracy, umożliwiającego zarówno poszukiwanie jej jak i poszukiwanie pracownika o odpowiednich kwalifikacjach.
Wynikiem projektu będzie w pełni funkcjonalny system mobilny złożony z wielu agentów, którzy będą reprezentować specjalistów oraz instytucje ich poszukujące. Będziemy rozpatrywać otoczenie branży IT.

\section{Opis problemu}
TODO\\ 
-dokończyć (opisac wysokopoziomowo co system bedzie robil)

Poszukiwanie pracy niesie ze sobą wiele powtarzalnych i żmudnych czynności, które można powierzyć inteligentnemu agentowi, którego zadaniem będzie odszukanie dostępnych ofert, przefiltrowanie i przekazanie użytkownikowi tylko tych, które są zgodne z podanymi warunkami.
Problem będziemy rozpatrywać z dwóch stron:
\begin{enumerate}
	\item poszukiwania pracy przez programistę,
	\item poszukiwania specjalistów spełniających kryteria rekrutacyjne.
\end{enumerate}

System ogłaszania oferty będzie oparty na tzw. `poczcie pantoflowej'. Oferty pracy będą przekazywane do pracowników i subskrybentów, którzy będą mogli przekazać ofertę innym agentom ze swojego otoczenia.


\subsection{poszukiwanie pracy przez programistę}
TODO\\
\subsection{poszukiwanie specjalistów}
TODO\\


\section{Specyfikacja funkcjonalna}
TODO\\
-szczegółowy opis problemu z wymienionymi funkcjonalnościami


\section{Techniczny opis systemu}
TODO\\
Potrzebne przemyslenie by zaczac klepac, ale do specyfikacji wstepnej sekcja nie musi wejsc.
podsekcje:
-uzyte technologie i co nam daja
-podział na moduły
-architektura


\section{Koncepcja eksperymentu- scenariusze testowe}
TODO\\
opis potwierdzony przykładami w jaki sposób mamy zamiar potwierdzić poprawne działanie systemu, czyli opi scenariuszy użycia od prostych po zaawansowane, np. (najprostszy przyklad) Javowiec poszukuje pracy na stanowisko senior developer, jest taka oferta na rynku i uzytkownik jest o niej informowany.

	
\end{document}